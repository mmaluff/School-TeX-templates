% !TEX spellcheck = en_US
\documentclass[letter,12pt]{article}
\usepackage{fancyhdr}
\usepackage{setspace}
\usepackage{fullpage}
\usepackage{amsmath}
\usepackage{amsfonts}
\usepackage{amssymb}
\usepackage{tikz}
\usepackage{tkz-euclide} 
\usetkzobj{all} 

\pagestyle{myheadings}
\markright{\hfill \textup{Maluff} }
\headsep 0.3in
\topmargin -0.3in
\textheight 9in

\newcommand{\N}{\mathbb{N}}
\newcommand{\Z}{\mathbb{Z}}
\newcommand{\Q}{\mathbb{Q}}
\newcommand{\R}{\mathbb{R}}
\newcommand{\C}{\mathbb{C}}
\newcommand{\F}{\mathbb{F}}

\doublespacing
\begin{document}

\begin{singlespace}
\noindent MATH \#\#\#: Geometry

\noindent Prof. Name Lastname

\noindent Mauricio Maluff Masi

\noindent 04/24/2013
\end{singlespace}
{\centering
\large{\textbf{Title}}

}
\vspace{12pt}

\noindent\textbf{1.} Give detailed instructions on how to construct a pentagon, using Euclid’s construction axioms. Be sure to explain your work at each stage.\\

Take any two points A and B. Draw the perpendicular bisector by the double-arc method, and call the midpoint O. Draw a circle with center O and radius OA.

\vspace{-20pt}


{\centering
    \begin{tikzpicture}[scale=4.0,cap=round,>=latex]
	\def\ptsize{0.6pt}

	\draw[thick] (0cm,0cm) circle(1cm);
	\draw[thick] (-1cm,0cm) -- (1cm,0cm);
	\draw[thick] (0cm,-1.2cm) -- (0cm,1.2cm);


	\coordinate (A) at (-1cm,0cm);
	\coordinate (B) at (1cm,0cm);
	\coordinate (M) at (0cm,1cm);
	\coordinate (O) at (0cm,0cm);

	\tkzShowLine[mediator, size=1, gap=1.2](A,B)

	\coordinate[label=left:A] (A)  at (A);
	\coordinate[label=right:B] (B)  at (B);
	\coordinate[label=above right:M] (M)  at (M);
	\coordinate[label=above right:O] (O)  at (O);


	\foreach \p in {A,B,M,O}
		\fill (\p) circle (\ptsize);


    \end{tikzpicture}

}



Then take the midpoint N of MO by the familiar method, and draw the line BN. Draw the angle bisector of $\widehat{\text{BNO}}$, and call D the intersection between the bisector and BO.























\end{document}
