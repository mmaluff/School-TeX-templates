% !TEX spellcheck = en_US
\documentclass[letter,12pt]{article}

%Packages
\usepackage{fullpage} % Nicer margins
\usepackage[svgnames]{xcolor} % Colors
\usepackage{minted} % Code syntax highlighting
\usepackage{enumitem} % Fancy enumerations
\usepackage{fontspec} % Fancy fonts

% Make my code look nice

% Set font used or code
\setmonofont{Ubuntu Mono}
% Pick the style for minted syntax highlighting
\usemintedstyle{xcode}
% Background color for minted
\definecolor{bg}{gray}{0.95}
% Default options for Java (use javacode environment to take advantage of it)
\newminted{java}{bgcolor=bg,linenos}
% Use \code{} for inline Java
\newcommand{\code}{\mintinline{java}}
% Add spaces around code blocks
\BeforeBeginEnvironment{minted}{\medskip}
\AfterEndEnvironment{minted}{\medskip}


% Header
\pagestyle{myheadings}
\markright{\hfill \textup{Maluff} }
\headsep 0.3in
\topmargin -0.3in
\textheight 9in

\begin{document}

\noindent CS \#\#\#: Class Name

\noindent Prof. Firstname Lastname

\noindent Mauricio Maluff Masi

\noindent 09/30/2016

{\centering
\large{\textbf{Homework 1}}

}
\vspace{12pt}
\noindent\textbf{Question 1:} Consider the following code fragment that is intended to remove every other item from a \code{List<String>} object. For example, given the list \code{[ "Ben", "Beck", "Eric",}
\code{"Vidya", "Jason", "Xiaoming" ]} it should remove \code{"Ben"}, \code{"Eric"}, and \code{"Jason"} and leave the list as just \code{[ "Beck", "Vidya", "Xiaoming" ]}.

\begin{javacode}
for (int i = 0; i < words.size(); i += 2) {
    words.remove(i); 
} 
\end{javacode}

\begin{enumerate}[label=\emph{\Alph*}.]
	\item First question
	\item Second question
	\item etc.
\end{enumerate}

\noindent\textbf{Answers}
\begin{enumerate}[label=\emph{\Alph*}.]
	\item First answer
	\item Second answer
	\item etc.
\end{enumerate}

\end{document}
