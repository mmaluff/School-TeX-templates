% !TEX spellcheck = en_US
\documentclass[letter,12pt]{article}
\usepackage{fancyhdr}
\usepackage{setspace}
\usepackage{fullpage}

\pagestyle{myheadings}
\markright{\hfill \textup{Maluff} }
\headsep 0.3in
\topmargin -0.3in
\textheight 9in

%7-page, double-spaced, TR 12-p, 1'' margins.

% Do a sustained critique of Rawls from one particular theoretical or political point of view (e.g. utilitarian, Kantian, libertarian, egalitarian, left-wing, right-wing, etc.), either trying to reconstruct his replies to you and showing why they don't work or showing why the critique doesn't work.

% Evaluated on: (a) accuracy of your representation of Rawls's views (b) the quality of your arguments.

\linespread{2}
\begin{document}

\begin{spacing}{1}
\noindent PHIL \#\#\#: Class name

\noindent Prof. Firstname Lastname

\noindent Mauricio Maluff Masi

\noindent 11/06/2012 \\

\end{spacing}

{\centering
\textbf{Title}

}

Lorem ipsum dolor sit amet, consectetur adipiscing elit, sed do eiusmod tempor incididunt ut labore et dolore magna aliqua. Ut enim ad minim veniam, quis nostrud exercitation ullamco laboris nisi ut aliquip ex ea commodo consequat. Duis aute irure dolor in reprehenderit in voluptate velit esse cillum dolore eu fugiat nulla pariatur. Excepteur sint occaecat cupidatat non proident, sunt in culpa qui officia deserunt mollit anim id est laborum.

{\centering
\textbf{Works cited}

}

Rawls, John. \emph{A Theory of Justice}. Cambridge, MA: Belknap of Harvard UP, 1971. Print.

\end{document}
